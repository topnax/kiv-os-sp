\documentclass[xcolor=dvipsnames]{beamer}
%\usepackage[utf8]{inputenc}
%\usepackage{xcolor}
\usepackage{graphicx}
\usepackage{tikz}
\usetikzlibrary{arrows,shapes}
\usepackage{caption}
%\usepackage[utf8]{inputenc}
\usepackage[czech]{babel}
%\usepackage[utf8]{vietnam}
\usepackage{pdfpages}
\usepackage{color}
\usepackage{booktabs}

%%%%%%%%%%%%%%%%%%%%%%%%%%%%%%%%%%%%%%%%%%%%%%%%%%%%%

%%%%%%%%%%%%%%%%%%%%%%%%%%%%%%%%%%%%%%%%%%%%%%%%%%%%%
%\usepackage{lipsum}


%%%%%%%%%%%%%%%%%%%%%%%%%%%%%%%%%%%%%%%%%%%%%%%%%%%%%

\usepackage{pgf}
\usepackage{etex}
\usepackage{tikz,pgfplots}

\tikzstyle{every picture}+=[remember picture]
% By default all math in TikZ nodes are set in inline mode. Change this to
% displaystyle so that we don't get small fractions.
\everymath{\displaystyle}


\usetheme{Madrid}
%\usetheme{Madrid}
%\usecolortheme[named=Maroon]{structure}
%\usecolortheme{crane}
\usefonttheme{professionalfonts}
\useoutertheme{infolines}
\useinnertheme{circles}

\newtheorem*{bem}{Bemerkung}

\usepackage{tikz}


%%%%%%%%%%%%%%%%%%%%%%%%%%%%%%%%%%%%%%%%%%%%%%%%%



%%%%%%%%%%%%%%%%%%%%%%%%%%%%%%%%%%%%%%%%%%%%%%%%%
%\usepackage{listings}
\usepackage{color}

\definecolor{dkgreen}{rgb}{1,0.6,0}
\definecolor{gray}{rgb}{1,1,0}
\definecolor{mauve}{rgb}{0.58,0,0.82}

%%%%%%%%%%%%%%%%%%%%%%%%%%%%%%%%%%%%%%%%%%%%%%%%%

\beamertemplatenavigationsymbolsempty 

\title[1. prezentace SP] %optional
{Simulace operačního systému}

\subtitle{1. prezentace semestrální práce z předmětu KIV/OS}

% \title[Západočeská univerzita v Plzni]{\includegraphics[width=\textwidth/4]{img/logo.png}}

\institute[ZČU FAV] % (optional)
{
    Západočeská univerzita v Plzni
    \and
    Fakulta aplikovaných věd
}

\author[Eliška, Ondřej, Stanislav] % (optional, for multiple authors)
{Eliška Mourycová, Ondřej Drtina, Stanislav Král}

%\logo{\includegraphics[height=1.5cm]{img/KIV\_ram\_cerna.pdf}}

\logo{\pgfimage[height=0.5cm]{img/kiv-logo.pdf}}


\begin{document}

\begin{frame}
  \titlepage
\end{frame}

\begin{frame}
\frametitle{Rozdělení bodů}
	\begin{itemize}
        \item \textbf{Eliška Mourycová} - 1/3 bodů
        \item \textbf{Ondřej Drtina} - 1/3 bodů
        \item \textbf{Stanislav Král} - 1/3 bodů
  	\end{itemize}
\end{frame}

\begin{frame}
\frametitle{Organizace projektu}
	\begin{itemize}
        \item umístěn na GitHub repozitáři
            \begin{itemize}
                \item využití systému úkolů (\textit{issues})
                \item 1 úkol = 1 větev (řešitel a reviewer)
                \item před mergem do masteru schválení PR (kontrola funkčnosti, formátování kódu, komentářů, \ldots)
            \end{itemize}
            \item stav repozitáře v současné chvíli:
            \begin{itemize}
                \item \textbf{18} otevřených issue 
                \item \textbf{12} uzavřených issue
                \item \textbf{8} schválených PR
                \item \textbf{1} čekající PR
            \end{itemize}
  	\end{itemize}
\end{frame}

\begin{frame}
\frametitle{Rozdělení práce na projektu}
    \begingroup
    \small
	\begin{itemize}
        \setlength\itemsep{0.1em}
        \item \textbf{Eliška Mourycová}
        \begin{itemize}
            \setlength\itemsep{0.1em}
            \item práce na programech v uživatelském prostoru (\texttt{freq}, argparser shellu, \textit{pipeline orchestration}, \ldots)
            \item kontrola a schvalování PR
        \end{itemize}
        \item \textbf{Ondřej Drtina}
        \begin{itemize}
            \setlength\itemsep{0.1em}
            \item studium FAT12 
            \item vytvoření PoC demonstrující práci s obrazem diskety
            \item vytvoření modulu pro práci se souborovým systémem
        \end{itemize}
        \item \textbf{Stanislav Král}
        \begin{itemize}
            \setlength\itemsep{0.1em}
            \item úvodní studium projektu (rozchození sestavení přes CMake$\,\to\,$\textbf{CLion})
            \item implementace programu \texttt{echo}
            \item implementace sys. volání z \texttt{Process} a obecná práce na kernelu
            \item implementace rour a tabulky souborů
        \end{itemize}
  	\end{itemize}
    \endgroup
\end{frame}

\begin{frame}[fragile]
\frametitle{Hotové úkoly}

\begin{itemize}
    \item všechna sys. volání ze skupiny \texttt{Process} (kromě \texttt{shutdown} sys. volání, blokováno \texttt{keyboard.cpp})
    \item programy \texttt{echo}, \texttt{rgen} a \texttt{freq}
    \item obecná tabulka souborů a její využití při IO
    \item objekt roury a její využití pro kombinaci \texttt{rgen} a \texttt{freq}
    \item parser pro příkazy shellu 
    \begin{verbatim}
        rgen | freq > freq_result.txt
    \end{verbatim}
    \item přečtení FAT tabulky na disketě
    \item přečtení souborů umístěných na disketě
    \item výpis souborů relativní k dané složce
\end{itemize}


\end{frame}

\begin{frame}[fragile]
\frametitle{Implementace \texttt{Wait\_For}}
    \begin{itemize}
        \item seznam semaforů, které čekají na daný \textit{handle}
        \item na začátku volání vytvořen nový semafor a přidán pro každý handle do seznamu semaforů
        \item při ukončení spuštěného vlákna/procesu notifikace semaforů
        \item původní idea: \textit{vlákno, které skončilo jistě probudilo semafor}
        \begin{itemize}
            \item zdá se jednoduché, ale řada nevýhod
            \item vyžaduje důkladnou ochranu KS a zamezení časového souběhu
        \end{itemize}
        \item při notifikaci také rovnou předávat informaci, který handle notifikoval
        \begin{itemize}
            \item \textit{uložit handle vedle semaforu, který notifikujeme}
        \end{itemize}   
    \end{itemize}
\end{frame}

\begin{frame}
\frametitle{Závěr}

\end{frame}

\end{document}

