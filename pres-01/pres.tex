\documentclass[xcolor=dvipsnames]{beamer}
%\usepackage[utf8]{inputenc}
%\usepackage{xcolor}
\usepackage[pdftex]{graphicx}
\usepackage{tikz}
\usetikzlibrary{arrows,shapes}
\usepackage{caption}
%\usepackage[utf8]{inputenc}
\usepackage[czech]{babel}
%\usepackage[utf8]{vietnam}
\usepackage{pdfpages}
\usepackage{color}
\usepackage{booktabs}

%%%%%%%%%%%%%%%%%%%%%%%%%%%%%%%%%%%%%%%%%%%%%%%%%%%%%

%%%%%%%%%%%%%%%%%%%%%%%%%%%%%%%%%%%%%%%%%%%%%%%%%%%%%
%\usepackage{lipsum}


%%%%%%%%%%%%%%%%%%%%%%%%%%%%%%%%%%%%%%%%%%%%%%%%%%%%%

\usepackage{pgf}
\usepackage{etex}
\usepackage{tikz,pgfplots}

\tikzstyle{every picture}+=[remember picture]
% By default all math in TikZ nodes are set in inline mode. Change this to
% displaystyle so that we don't get small fractions.
\everymath{\displaystyle}


\usetheme{Madrid}
%\usetheme{Madrid}
%\usecolortheme[named=Maroon]{structure}
%\usecolortheme{crane}
\usefonttheme{professionalfonts}
\useoutertheme{infolines}
\useinnertheme{circles}

\newtheorem*{bem}{Bemerkung}

\usepackage{tikz}


%%%%%%%%%%%%%%%%%%%%%%%%%%%%%%%%%%%%%%%%%%%%%%%%%



%%%%%%%%%%%%%%%%%%%%%%%%%%%%%%%%%%%%%%%%%%%%%%%%%
%\usepackage{listings}
\usepackage{color}

\definecolor{dkgreen}{rgb}{1,0.6,0}
\definecolor{gray}{rgb}{1,1,0}
\definecolor{mauve}{rgb}{0.58,0,0.82}

%%%%%%%%%%%%%%%%%%%%%%%%%%%%%%%%%%%%%%%%%%%%%%%%%

\beamertemplatenavigationsymbolsempty 

\title[1. prezentace SP] %optional
{Simulace operačního systému}

\subtitle{1. prezentace semestrální práce z předmětu KIV/OS}

% \title[Západočeská univerzita v Plzni]{\includegraphics[width=\textwidth/4]{img/logo.png}}

\institute[ZČU FAV] % (optional)
{
    Západočeská univerzita v Plzni
    \and
    Fakulta aplikovaných věd
}

\author[Eliška, Ondřej, Stanislav] % (optional, for multiple authors)
{Eliška Mourycová, Ondřej Drtina, Stanislav Král}

%\logo{\includegraphics[height=1.5cm]{img/KIV\_ram\_cerna.pdf}}

\logo{\pgfimage[height=0.5cm]{img/kiv-logo.pdf}}


\begin{document}

\begin{frame}
  \titlepage
\end{frame}

\begin{frame}
\frametitle{Rozdělení bodů}
	\begin{itemize}
        \item \textbf{Eliška Mourycová} - 1/3 bodů
        \item \textbf{Ondřej Drtina} - 1/3 bodů
        \item \textbf{Stanislav Král} - 1/3 bodů
  	\end{itemize}
\end{frame}

\begin{frame}
\frametitle{Organizace projektu}
	\begin{itemize}
        \item umístěn na GitHub repozitáři
            \begin{itemize}
                \item využití systému úkolů (\textit{issues})
                \item 1 úkol = 1 větev (řešitel a reviewer)
                \item před mergem do masteru schválení PR (kontrola funkčnosti, formátování kódu, komentářů, \ldots)
            \end{itemize}
            \item stav repozitáře v současné chvíli:
            \begin{itemize}
                \item \textbf{18} otevřených issue 
                \item \textbf{12} uzavřených issue
                \item \textbf{8} schválených PR
                \item \textbf{1} čekající PR
            \end{itemize}
  	\end{itemize}
\end{frame}

\begin{frame}
\frametitle{Rozdělení práce na projektu}
    \begingroup
    \small
	\begin{itemize}
        \setlength\itemsep{0.1em}
        \item \textbf{Eliška Mourycová}
        \begin{itemize}
            \setlength\itemsep{0.1em}
            \item práce na programech v uživatelském prostoru (\texttt{freq}, argparser shellu, \textit{pipeline orchestration}, \ldots)
            \item kontrola a schvalování PR
        \end{itemize}
        \item \textbf{Ondřej Drtina}
        \begin{itemize}
            \setlength\itemsep{0.1em}
            \item studium FAT12 
            \item vytvoření PoC demonstrující práci s obrazem diskety
            \item vytvoření modulu pro práci se souborovým systémem
        \end{itemize}
        \item \textbf{Stanislav Král}
        \begin{itemize}
            \setlength\itemsep{0.1em}
            \item úvodní studium projektu (rozchození sestavení přes CMake\footnote{\url{https://github.com/pavelliavonau/cmakeconverter}}$\,\to\,$\textbf{CLion})
            \item implementace programu \texttt{echo}
            \item implementace sys. volání z \texttt{Process} a obecná práce na kernelu
            \item implementace rour a tabulky souborů
        \end{itemize}
  	\end{itemize}
    \endgroup
\end{frame}

\begin{frame}[fragile]
\frametitle{Hotové úkoly}

\begin{itemize}
    \item všechna sys. volání ze skupiny \texttt{Process} (kromě \texttt{shutdown} sys. volání, blokováno \texttt{keyboard.cpp})
    \item programy \texttt{echo}, \texttt{rgen} a \texttt{freq}
    \item obecná tabulka souborů a její využití při IO
    \item objekt roury a její využití pro kombinaci \texttt{rgen} a \texttt{freq}
    \item parser pro příkazy shellu 
    \begin{verbatim}
        rgen | freq > freq_result.txt
    \end{verbatim}
    \item přečtení FAT tabulky na disketě
    \item přečtení souborů umístěných na disketě
    \item výpis souborů relativní k dané složce
\end{itemize}


\end{frame}

\begin{frame}[fragile]
\frametitle{Implementace \texttt{Wait\_For}}
    \begin{itemize}
        \item seznam semaforů, které čekají na daný \textit{handle}
        \item na začátku volání vytvořen nový semafor a přidán pro každý handle do seznamu semaforů
        \item při ukončení spuštěného vlákna/procesu \textbf{notifikace semaforů}
        \item původní idea: \textit{vlákno, které skončilo, jistě probudilo semafor}
        \begin{itemize}
            \item zdá se jednoduché, ale řada nevýhod
            \item vyžaduje důkladnou ochranu KS a zamezení časového souběhu
        \end{itemize}
        \item při notifikaci také rovnou \textbf{předávat informaci}, který handle notifikoval
        \begin{itemize}
            \item \textit{uložit handle vedle semaforu, který notifikujeme}
        \end{itemize}   
    \end{itemize}
\end{frame}

\begin{frame}[fragile]
\frametitle{Implementace souborové tabulky}
    \begin{itemize}
        \item původní návrh byl inspirován Unixem (3 tabulky)
        \begin{itemize}
            \item byl částečně implementován
            \item z důvodu složitosti ale \textbf{zahozen}
        \end{itemize}
        \item nakonec implementováno pouze jednou tabulkou (\texttt{std::map})
        %\item před spuštěním shellu otevřeny soubory \texttt{/dev/vga} a \texttt{/dev/keyboard}
    \end{itemize}
    \begin{table}[]
    \begin{tabular}{|l|l|}
        \hline
        \textbf{THandle} & \textbf{Generic\_File}   \\ \hline \hline
        0       & \texttt{Pipe\_In\_File}  \\ \hline
        1       & \texttt{Pipe\_Out\_File} \\ \hline
        2       & \texttt{Vga\_File}       \\ \hline
    \end{tabular}
    \end{table}       
    \begin{itemize}
        \item do tabulky se přidávají položky pomocí sys. volání \texttt{Open\_File}
    \end{itemize}
\end{frame}

\begin{frame}[fragile]
\frametitle{Třída \texttt{Generic\_File}}
    \begin{itemize}
        \item abstraktní třída představující obecný soubor (\textit{file object})
        \item její instance \textbf{uloženy v tabulce souborů}
        \item definuje virtuální metody \texttt{write}, \texttt{read} a \texttt{close} 
        \item vytvoření potomci: \texttt{Vga\_File}, \texttt{Keyboard\_File} a \texttt{Pipe\_In\_File}/\texttt{Pipe\_Out\_File}
        \item \textbf{před spuštěním} shellu vytvořeny instance \texttt{Vga\_File} a \texttt{Keyboard\_File}
        \item musí definovat virtuální destruktor (jinak nefunkčnost \texttt{std::unique\_ptr})
    \end{itemize}
        \begingroup
        \small
        \begin{verbatim}        
   
    // deleting child class instances invokes child destructor
    virtual ~Generic_File() = default;
        \end{verbatim}
       \endgroup
\end{frame}

\begin{frame}
\frametitle{Implementace rour}
    \begin{itemize}
        \item problém \textbf{producent/konzument}
        \item instance třídy \texttt{Pipe} obalovaná souborovými objekty \texttt{Pipe\_In\_File}/\texttt{Pipe\_Out\_File}
        \item v třídě \texttt{Pipe}
        \begin{itemize}
            \item řešení problému P\&K \textbf{pomocí semaforů}
            \item upravené tak, aby šlo zapsat/přečíst najednou \texttt{n} bytů
            \item buffer (64k) pomocí \texttt{std::vector} + 2 ukazatele (\texttt{read} a \texttt{write})
        \end{itemize}
        \item možnost uzavření vstupu/výstupu (\texttt{EOT}) 
    \end{itemize}
    \begin{figure}[!ht]
    \centering
    {\includegraphics[width=0.85\textwidth]{pdf/pipe.pdf}}
    \caption{Diagram obalení roury.}
    \label{fig:screen-transition-diagram}
    \end{figure}
\end{frame}

%\begin{frame}
%\frametitle{Použití argparseru a orchestrace rour}
%    \begin{itemize}
%        \item shell musí umět chytře spouštět programy tak, aby šlo použít operátorů \texttt{>}, \texttt{<} a \texttt{|} 
%        \item vytvořit potřebné množství rour
%        \item před spuštěním nastavit správně \texttt{std\_in} a \texttt{std\_out}
%        \item iterativně volat \texttt{Wait\_For} na spuštěné procesy a uzavírat roury/\textit{file handly}
%    \end{itemize}
%\end{frame}

\begin{frame}
\frametitle{Závěr}
    \begin{itemize}
        \item značná část práce již hotova 
        \item dokončené v nejbližší době bude
        \begin{itemize}
            \item použití argparseru a orchestrace rour
            \item výpis obsahu složky 
            \item systémové volání  \texttt{shutdown}
        \end{itemize}
    \end{itemize}
\end{frame}

\begin{frame}
\center
Děkuji za pozornost,
nyní je prostor pro otázky
\end{frame}

\end{document}

